\documentclass[a4paper,14pt]{article}
\usepackage[colorlinks, linkcolor=blue]{hyperref}
\usepackage{indentfirst}
\usepackage[T1,T2A]{fontenc}
\usepackage[utf8]{inputenc}
\usepackage[russian]{babel}
\usepackage[fleqn]{amsmath}
\usepackage{amssymb}
\usepackage{graphicx}
\usepackage{float}
\usepackage{sectsty}
\usepackage{fancyhdr}
\usepackage{titling}
\usepackage{datetime}
\usepackage{amsthm}
\usepackage[headheight=15pt]{geometry}
\usepackage{pstricks}
\usepackage{color}
\usepackage[linesnumbered, noline, noend]{algorithm2e}
\usepackage{caption}
\usepackage{subcaption}
\usepackage{verbatim}
\usepackage{amsfonts}
\usepackage{epstopdf}
\usepackage{mathtools}

%opening
\title{Индивидуальное задание 1\\
	``Нерекурсивный и рекурсивный фильтр''\\
	Вариант 7}
\author{Илья Мурадьян, группа 4.1}

\sectionfont{\normalfont\centering\textbf}
\geometry{left=2cm}% левое поле
\geometry{right=1.5cm}% правое поле
\geometry{top=1cm}% верхнее поле
\geometry{bottom=2cm}% нижнее поле

\let\oldref\ref
\renewcommand{\ref}[1]{(\oldref{#1})}

\begin{document}
	\Large
	
	\maketitle
	
	Мой вариант предполагал использование следующих данных для фильтра и входного сигнала:
	\par \vspace{0.3cm}
	
	\bgroup
	\def\arraystretch{1.5}
	\begin{tabular}{|c|c|c|c|c|c|}
		\hline
		$a_1$ & $a_2$ & $a_3$ & $b_0$ & $b_1$ & $b_2$ \\
		\hline
		$-\frac{4}{3}$ & $-\frac{7}{12}$ & $-\frac{1}{12}$ & $1$ & $-1$ & $-2$ \\
		\hline
	\end{tabular}
	\egroup
	\par \vspace{0.3cm}
	\begin{tabular}{|c|c|c|c|c|c|c|c|c|}
		\hline
		$x_0$ & $x_1$ & $x_2$ & $x_3$ & $x_4$ & $x_5$ & $x_6$ & $x_7$ & $x_8$ \\
		\hline
		$3$ & $5$ & $-2$ & $-4$ & $4$ & $2$ & $-1$ & $-4$ & $2$ \\
		\hline
	\end{tabular}

	\vspace{0.3cm}
	Рекурсивный фильтр задан следующим соотношением:
	\begin{equation} \label{eq:recfilt}
		y_n = a_1 y_{n-1} + a_2 y_{n-2} + a_3 y_{n-3} + b_0 x_n + b_1 x_{n-1} + b_2 x_{n-2}
	\end{equation}
	
	Чтобы найти импульсную характеристику сигнала, положим 
	\[x_n = \delta_n = \begin{cases}
	1, & n = 0 \\
	0, & n \ne 0 \\
	\end{cases} \]
	
	Тогда при $n \ge 3$ получим:
	\begin{equation} \label{eq:rfilt}
	h_n = a_1 h_{n-1} + a_2 h_{n-2} + a_3 h_{n-3}.
	\end{equation}
	
	С нашими числовыми данными имеем:
	\[h_n = -\frac{4}{3} h_{n-1} -\frac{7}{12} h_{n-2} -\frac{1}{12} h_{n-3}\]
	\[12 h_n + 16 h_{n-1} + 7 h_{n-2} + h_{n-3} = 0\]
	
	Для этого разностного уравнения выпишем характеристический многочлен и найдём его корни:
	\[L(\lambda) = 12 \lambda^3 + 16 \lambda^2 + 7 \lambda + 1\]
	\[ \lambda_{1, 2} = -\frac{1}{2}, \lambda_{3} = -\frac{1}{3} \]
	
	Оба найденных корня по модулю меньше единицы, так что исследуемая БИХ-система устойчива. Решение ищем в виде:
	\begin{equation} \label{eq:solv}
		h_n = (C_1 + C_2 n) \left ( -\frac{1}{2} \right )^n + C_3 \left (-\frac{1}{3}\right )^n 
	\end{equation}
	
	Подставляя в уравнение \eqref{eq:rfilt} единичный сигнал при $n = 0, 1, 2$ имеем:
	\begin{equation}
	\begin{cases} \label{eq:hsyst}
	h_0 = b_0\\
	h_1 = a_1 h_0 + b_1 = a_1 b_0 + b_1\\
	h_2 = a_1 h_1 + a_2 h_0 + b_2 = a_1^2 b_0 + a_1 b_1 + a_2 b_0 + b_2
	\end{cases}
	\end{equation}
	
	Подставим теперь в левую часть системы \eqref{eq:hsyst} общий вид решения \eqref{eq:solv}, а в правую часть -- исходные данные, и получим следующую линейную систему относительно $C_1, C_2, C_3$:
	\begin{equation}
		\begin{pmatrix}
		1 & 0 & 1 \\
		-\frac{1}{2} & -\frac{1}{2} & -\frac{1}{3}\\
		\frac{1}{2} & \frac{1}{2} & \frac{1}{9}\\
		\end{pmatrix}
		\begin{pmatrix}
		C_1 \\
		C_2 \\
		C_3 \\
		\end{pmatrix}
		=
		\begin{pmatrix}
		1 \\
		-\frac{7}{3}\\
		\frac{19}{36}\\
		\end{pmatrix}
	\end{equation}
	
	Решая эту систему, получаем:
	\begin{equation}
	C_1 = 57, C_2 = -15, C_3 = -56.
	\end{equation}
	
	Подставим теперь найденные значения констант в \eqref{eq:solv} и получим окончательную формулу для импульсной характеристики:
	\begin{equation}
	h_n = (57 - 15 n) \left ( -\frac{1}{2} \right )^n -56 \left (-\frac{1}{3}\right )^n 
	\end{equation}
	
	По полученной формуле вычислим $h_5$:
	\[h_5 \approx 0.79295267489712\]
	
	Такое же значение получается при применении фильтра, что заставляет нас думать, что формула была найдена верно.
	
	Рассмотрим теперь следущие два финитных сигнала:
	\begin{equation}
		\widetilde{x}_n = \begin{cases}
		x_n, & 0 \le n \le 8 \\
		0, & \textup{иначе} \\
		\end{cases}
	\end{equation}
	\begin{equation}
		\widehat{x}_n = \begin{cases}
		x_n + \frac{1}{10}, & 0 \le n \le 8 \\
		\frac{1}{10}, & n \in \{ -1, 9 \} \\
		0, & \textup{иначе} \\
		\end{cases}
	\end{equation}
	
	Найдём тысячный отсчет отклика системы на каждый из этих сигналов:
	\[ \widetilde{y}_{1000} \approx -1,352278 \cdot 10^{-294}. \]
	\[ \widehat{y}_{1000} \approx -1,305085 \cdot 10^{-294}. \]
	
	Как видно, порядки и первые две значащие цифры полученных результатов совпадают. Это говорит о хорошей устойчивости системы.
	
	Теперь найдём несколько значений импульсной характеристики: \newline
	\vspace{0.3cm}
	\begin{tabular}{|c|c|c|c|c|c|c|c|c|c|}
		\hline
		$h_0$ & $h_1$ & $h_2$ & $h_3$ & $h_4$ & $h_5$ & $h_6$ & $h_7$ & $h_8$ & $h_9$ \\
		\hline
		$1,000$ & $-2,333$ & $0,527$ & $0,574$ & $-0,879$ & $0,793$ & $-0,592$ & $0,401$ & $-0,255$ & $0,155$ \\
		\hline
	\end{tabular}
	
	\vspace{0.3cm}
	Построим следущий фильтр с конечной импульсной характеристикой: 
	\begin{equation*}
	\begin{array}{l}
	y_n^{'} = \sum\limits_{k \in \mathbb{Z}} x_k \widehat{h}_{n - k},  \\
	\textup{где} \quad \widehat{h}_{i} = \begin{cases}
	h_i, & 0 \le i \le 9 \\
	0, & \textup{иначе} \\
	\end{cases} \\
	\end{array}
	\end{equation*}
	
	Посчитаем теперь 1000-й отсчёт отклика построенного фильтра на сигнал $ \widetilde{x} $:
	\[ \widetilde{y}_{1000}^{'} = 0.\]
	
	Начиная с некоторого $n$ все отсчёты отклика построенного фильтра на заданный финитный сигнал будут нулевыми. Так получилось и сейчас. Однако с учётом того, что тысячный отсчёт фильтра \eqref{eq:recfilt} ничтожно мал, можно назвать это приближение удовлетворительным.
	
	Работа была выполнена мною лично, без чьей-либо помощи, без использования нелицензионных программ.
\end{document}
